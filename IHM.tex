\documentclass[11pt]{article}

\usepackage[utf8]{inputenc}
\usepackage[x11names]{xcolor}   % Accès à une table de 317 couleurs
\usepackage{graphicx} %Pur utiliser la colorbox
\usepackage{textcomp}
\usepackage{amsmath}
\usepackage{amssymb}

\title{\textbf{IHM \\}}
\author{RETAIL Tanguy}
\date{26/01/2016}
\begin{document}

\maketitle
\tableofcontents
\newpage


\section{Enoncé}
\textbf{Interface tactile pour un restaurant}\\
Un restaurant vous approche demandant de créer une interface numérique pour mettre à disposition des clients dans un restaurant pour prendre les commandes.\\
Vous allez faire une récolte de besoins et analyse des tâches.\\
Créer plusieurs prototypes basse-fidélités.\\
Construire un prototype haute-fidélité.\\

\section{Restaurant}
\subsection{Idées en vrac}
Restaurant chic, ou tout du moins design/branché, donc l'installation peut se justifier pour le design\\
une tablette/personne : possibilité de différencier les menus de chacun\\
- identification de chaque place sur chaque table\\
Un rangement amovible adapté à la tablette pour permettre de sortir/ranger la tablette au besoin\\
\section{Tâche-Utilisateur-Technologie}
\subsection{Utilisateurs}
\subsubsection{Client}
Moyenne d'âge visée : 20 à 45ans \\
Connaissance de la technologie : hétérogène, aucune connaissance à beaucoup.\\

\subsubsection{Serveur}
Connaissance de la technologie : a subi une formation avant.\\

\subsubsection{Cuisinier}
Connaissance de la technologie : a subi une formation avant.\\

\subsubsection{Trésorier, Manager}
Connaissance de la technologie : a subi une formation avant.\\

Profils\\
Quels attributs pertinents ?\\

\subsection{Tâche}
\textbf{Quelles sont-elles ?}\\
\subsubsection{Client}
- parcours de menu\\
- gestion de panier (commande): \\
         * choix\\
         * modification\\
- demande d'addition (avec choix de moyen de paiement)\\
- appel d'un serveur pour toute autre demande:\\
         * aide à la commande\\
         * couverts\\
         * précisions sur les ingrédients\\
         * réclamations, ...\\
Cagnotte commune ? Payent-ils ensemble ? Possible de mettre en réseau

\subsubsection{Serveur}
- prise de commande avec identification de table\\
- validation de commande délivrée\\
- validation des commandes payées\\
- reception des notifications\\
\subsubsection{Cuisinier}
- notification des commandes prêtes\\
- acceptation de commandes\\
- impossibilité d'assurer une commande\\
\subsubsection{Trésorier, Manager}
- connaître les rentrées d'argent à chaque instant\\
- savoir quels plats sont commandés, en quelle quantité ?\\
- modification des cartes des menus.\\

\end{document}